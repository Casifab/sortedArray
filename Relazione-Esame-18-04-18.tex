\documentclass[11pt]{article}
\usepackage[utf8x]{inputenc}
\usepackage[italian]{babel}
%\usepackage[T1]{fontenc}
\usepackage{hyperref}
\author{Fabio Casiraghi \\807398 \\ \href{mailto:f.casiraghi@campus.unimib.it} {f.casiraghi@campus.unimib.it}}
\title{Relazione per l'esame di Programmazione e Amministrazione di Sistema}
\begin{document}
\maketitle
\vspace{2cm}

\section*{Scelte implementative}
Per la realizzazione del sorted array templato, non era possibile mantenere in memoria
due copie simultanee dello stesso array. \\
Ho quindi optato per l'uso di un normale array di elementi T per mantenere l'ordine di memorizzazione,
mentre un array di puntatori a elementi T per mantenere l'ordine logico. Questa scelta permette
una gestione più efficiente dell'inserimento e della ricerca ordinata di un elemento all'interno della
struttura dati.\\
Per l'inserimento, infatti, è necessario semplicemente scorrere l'array di puntatori per trovare la 
posizione adatta al nuovo elemento e successivamente aggiornare i puntatori da quella posizione fino al fondo
della struttura. Il tutto può migliorare significativamente i tempi nel caso di un array 
con un numero molto elevato di elementi.\\
Per la ricerca ordinata, invece, basterà semplicemente scorrere l'array di puntatori finchè non viene 
trovato l'elemento desiderato. Questo metodo rispecchia fedelmente le normali procedure di ricerca e inserimento 
in un array.
\section*{Differenze con i tradizionali array}
\end{document}