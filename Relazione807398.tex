\documentclass[11pt]{article}
\usepackage[utf8x]{inputenc}
\usepackage[italian]{babel}
\usepackage[T1]{fontenc}
\usepackage{hyperref}
\author{Fabio Casiraghi \\807398 \\ \href{mailto:f.casiraghi@campus.unimib.it} {f.casiraghi@campus.unimib.it}}
\title{Relazione per l'esame di Programmazione e Amministrazione di Sistema}
\begin{document}
\maketitle
\vspace{2cm}

\section*{Scelte implementative}
Per la realizzazione del sorted array templato, non era possibile mantenere in memoria
due copie simultanee dello stesso array. \\
Ho quindi optato per l'uso di un normale array di elementi T per mantenere l'ordine di memorizzazione,
mentre un array di puntatori a elementi T per mantenere l'ordine logico. Questa scelta permette
una gestione più efficiente dell'inserimento e della ricerca ordinata di un elemento all'interno della
struttura dati.\\
Per l'inserimento, infatti, è necessario semplicemente scorrere l'array di puntatori per trovare la 
posizione adatta al nuovo elemento e successivamente aggiornare i puntatori da quella posizione fino al fondo
della struttura. Il tutto può migliorare significativamente i tempi nel caso di un array 
con un numero molto elevato di elementi.\\
Per la ricerca ordinata, invece, basterà semplicemente scorrere l'array di puntatori finchè non viene 
trovato l'elemento desiderato. Questo metodo rispecchia fedelmente le normali procedure di ricerca e inserimento 
in un array.\\
All'interno della classe vengono lanciate eccezioni solamente in presenza di new che potrebbero portare a 
errori nell'allocazione della memoria. Per l'inserimento di un elemento in un array già pieno, invece, si 
è pensato fosse più adatta una stampa a schermo, per avvisare esplicitamente l'utente del mancato caricamento di 
uno o più dati.\\
All'interno della classe vengono lanciate e gestite eccezioni all'interno dei metodi dove avviene un'esplicita allocazione di memoria (tramite l'operatore new). Lo scopo è quello di evitare errori di gestione della memoria, 
mentre l'utente utilizza la libreria in maniera conforme al cuo scopo.
\section*{Differenze con i tradizionali array}
La sostanziale differenza tra il sortedarray e uno tradizionale è l'impossibilità di scrivere dati in posizioni specifiche dell'array. Infatti sia gli operatori ( ) e [ ] che i due iteratori sono costanti e in sola lettura, inoltre i nuovi elementi vengono sempre aggiunti in coda nell'array per memorizzazione e successivamente vengono aggiornati i puntatori nell'array per l'ordine logico.\\
Questa scelta è dovuta principalmente all'idea che un utilizzatore della struttura dati non possa sovrascrivere dati in posizioni random, rompendo così sia l'ordinamento logico che l'ordinamento secondo memorizzazione.\\
Seguendo la logica di cui sopra, non è nemmeno possibile la rimozione dalla memoria di dati in posizione specifiche, ma è solo possibile svuotare completamente la struttura dati.\\
Come durante l'utilizzo di un normale array, il tentativo di accesso a locazioni di memoria adiacenti a quelle 
allocate per il sorted\_array genera degli errori a run-time, che tuttavia non vengono gestiti dalla libreria 
tramite il lancio di eccezioni. Si lascia infatti che vengano lanciati i messaggi d'errore del compilatore che 
solitamente appaiono per i tradizionali array (quali effettivamente sono nd\_array e pos\_array).
\section*{Const-iterator e Unsorted-const-iterator}
Per la scelta del tipo dei due iteratori si è pensato che il random iterator fosse il più adatto allo scopo. Infatti per rendere il più possibile l'utilizzo della struttura simile ad un array tradizionale, è necessario che si possano creare puntatori ad elementi in posizioni specifiche del sortedarray.\\
Entrambi gli iteratori sono costanti, e quindi in sola lettura. Il motivo è il medesimo della sezione precedente, si vuole impedire all'utente di modificare la correttezza dell'ordinamento di memorizzazione o logico.\\
Per evitare collisione tra i metodi begin( ) e end( ) dei due iteratori, che avrebbero avuto la medesima signature, 
si è deciso di nominare i metodi, per l'unsorted-const-iterator, ubegin( ) e uend( ).\\
Data la sostanziale differenza logica tra i due tipi di iteratori, non è sembrato necessario sviluppare dei metodi 
per il confronto o la conversione tra tipi diversi.\\
All'interno degli iteratori non vengono generate eccezioni, non essendoci allocazioni tramite new o altre manipolazioni della memoria. L'iterazione su elementi allocati ma non inizializzati o al di fuori della memoria 
allocata per il sorted\_array potrebbe generare errori a run-time o rilevabili da valgrind. Non viene considerato 
obiettivo della libreria gestire questi possibili errori, poichè si considera buona pratica evitare, solitamente, di 
accedere a locazioni di memoria non esplicitamente allocate. Il dovere di evitare la generazione di questi errori 
viene quindi lasciata all'utente. 
\section*{Breve descrizione dei file}
\subsection*{sortedarray.h}
Contiene la definizione dell'array templato, elemento centrale del progetto. Il file contiene la classe 
sorted\_array con all'interno le definizioni dei due iteratori, dichiarati friend della classe superiore, in 
modo che essa possa utilizzare i suoi metodi privati per la costruzione di iteratori.\\
Oltre alla classe sorted\_array sono presenti le definizioni di tre funzioni globali:
\begin{itemize}
\item \textbf{find\_count} funzione che stampa a schermo quanti elementi T del sorted\_array soddisfano il predicato 
passato come parametro, applicato tra l'elemento del sorted\_array e quello T passato come parametro alla funzione. 
La funzione scorre in ordine logico tutti gli elementi memorizzati all'interno dell'array e tiene conto di quello che generano una risposta booleana positiva. Restituisce una stringa con il conto dei predicati.
\item \textbf{operator<<} operatore di ostream per il sorted\_array, che stampa i dati in ordine logico.
\item \textbf{printMem} funzione che vuole emulare un operatore di ostrem, per poter stampare con una sola istruzione 
i dati anche in ordine di memorizzazione
\end{itemize}
\subsection*{main.cpp}
File contenente i principali casi di test per la classe sorted\_array. Vengono inoltre usate struct custom presenti nel file dati\_prova.h.
\subsection*{dati\_prova.h}
File contenente la definizione della struct custom persona, che rappresenta la persona tramite due stringhe, una per 
il nome e una per il cognome, un intero che simboleggia l'età e un valore booleano che rappresenta il sesso 
(1 se femmina, 0 se maschio). Definizioni di funtori per il confronto sia tra dati di tipo standard che custom.
\section*{Strumenti di sviluppo utilizzati}
\begin{itemize}
\item \textbf{G++} versione 5.4.0 per Linux (Mint 18.3 Sylvia)
\item \textbf{Valgrind} versione 3.11.0
\item \textbf{GNU make} versione 4.1
\item \textbf{Doxygen} versione 1.8.11
\end{itemize} 
\end{document}